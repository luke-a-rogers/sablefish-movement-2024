\documentclass{article}
\usepackage[utf8]{inputenc}
\usepackage[margin=2cm]{geometry}
\usepackage{amsmath, amssymb}
\usepackage{bm}
\usepackage{gensymb}
\usepackage{lineno}
\usepackage{lscape}
\renewcommand\linenumberfont{\normalfont\bfseries\small\color{darkgrey}}
\usepackage{booktabs}
\usepackage[round]{natbib}
\bibliographystyle{plainnat}
\usepackage{authblk}
% Linux Libertine:
\usepackage{textcomp}
\usepackage[sb]{libertine}
\usepackage[varqu,varl]{inconsolata}% sans serif typewriter
\usepackage[libertine,bigdelims,vvarbb]{newtxmath} % bb from STIX
\usepackage[cal=boondoxo]{mathalfa} % mathcal
\useosf % osf for text, not math
\usepackage{setspace}
\usepackage{siunitx}
\usepackage[section]{placeins} % Keep floats (figs, tabs, eqns) in the right section
% I assume these all work together to make nice colour
\usepackage[dvipsnames]{xcolor}
\definecolor{niceblue}{HTML}{236899} % Depends \usepackage[dvipsnames]{xcolor}
\definecolor{darkgrey}{HTML}{A9A9A9} % Depends \usepackage[dvipsnames]{xcolor}
% End nice colour
\usepackage{pgfplotstable} % For tables
\pgfplotsset{compat=1.16} % For tables
\usepackage{graphicx}
\graphicspath{ {./figs/} }
% hyperref must be last in preamble
\usepackage{hyperref}
\hypersetup{
    colorlinks=true,
    linkcolor=black,
    filecolor=black,      
    urlcolor=niceblue,
    citecolor=niceblue,
    linkbordercolor = white
}

% Title
\title{Sablefish equations }

% Authors
\author[1]{Luke A. Rogers}
\author[]{...}

%Affiliations
\affil[1]{Pacific Biological Station, Fisheries and Oceans Canada, Nanaimo, BC, V9T 6N7, Canada}

\begin{document}

\maketitle
\linenumbers
\setcounter{secnumdepth}{0}

\section{Notation: decision point}

Briefly: I propose two options for index notation---compact and verbose---and I advocate for compact.
\newline

\noindent Background: The largest model arrays have 5 dimensions that correspond to 5 index slots. Two slots are spatial, two are temporal, and one is size.
\newline

\noindent Details: The compact option encodes the two spatial indexes in one vector $\boldsymbol{s} = \left( s_0,s \right)$ and the two temporal indexes in another vector $\boldsymbol{t} = \left( t_0, t \right)$ wherever possible. This reduces the number of subscript slots displayed in the equations.
\newline

\noindent Tradeoffs: The compact option uses more symbols (vectors, scalars) and faces (bold, plain) in the subscripts, but never uses more than three subscript slots. The verbose option uses fewer symbols and only one face (plain) in the subscripts, but uses up to five subscript slots.
\newline

\noindent Example:
% Example
\begin{align}
  \mathrm{Compact:}\qquad \quad y_{\boldsymbol{s}, \boldsymbol{t},l} &\sim \mathrm{NegBinom2} \!
                                                                 \left[\mu_{\boldsymbol{s}, \boldsymbol{t},l}
                                                                 \mathrm{,} \: \phi \right] \\
  \mathrm{Verbose:}\qquad y_{s_0,s,t_0,t,l} &\sim \mathrm{NegBinom2} \!
                                              \left[\mu_{s_0,s,t_0,t,l} \mathrm{,} \: \phi \right]
\end{align}

\noindent Upshot: I recommend the compact notation because it is concise. I believe it gives a better sense of the model on a quick scan, and a more direct route to understanding the nuances on a deep dive.

\newpage

\section{Movement model}

\noindent Indexes: The index notation encodes the two spatial indexes in one vector $\boldsymbol{s} = \left( s_0,s \right)$ and the two temporal indexes in another vector $\boldsymbol{t} = \left( t_0, t \right)$ wherever possible. The components are shown explicitly where they are needed. Shorthands for the initial conditions $\boldsymbol{s}_0 = (s_0,s_0)$ and $\boldsymbol{t}_0 = (t_0,t_0)$ indicate the spatial and temporal indexes corresponding to a tag release event.
\newline

\noindent Symbols: Bold face symbols (other than indexes) correspond to square (sometimes diagonal) matrix slices of larger arrays. Matrix rows correspond to initial or previous spatial regions, while columns correspond to current regions. Consequently, the first two subscript index slots ($\boldsymbol{s} = (s_0, s)$) are suppressed for matrices. Matrix location within an array is given by the values in the remaining subscript index slots. 

Plane face symbols are scalars (often elements of arrays). Matrix symbols are used except where values are inherently univariate, for example the scalar outcome of a univariate sampling distribution. Scalar membership in an array is identified by symbol letter (greek or latin) and letter case. Consequently, different letter cases (e.g. $\boldsymbol{\lambda}_{l}$ and $\boldsymbol{\Lambda}_{t,l}$) correspond to different arrays.

The $\mathrm{diag} \! \left[ \cdot \right]$ function constructs a vector from the diagonal of a square (and diagonal) matrix.


% Sampling model
\begin{equation}
  \label{eq:model-sampling}
  y_{\boldsymbol{s}, \boldsymbol{t},l} \sim \mathrm{NegBinom2} \!
    \left[\mu_{\boldsymbol{s}, \boldsymbol{t},l} \mathrm{,} \: \phi \right]
\end{equation}

% % Observation model
% \begin{equation}
%   \label{eq:model-observation}
%   \widehat{y}^{\:\mathrm{obs}}_{a,b,c,d,e} \sim 
%     \mathrm{Distrn} \! \left[\widehat{y}^{\:\mathrm{proc}}_{a,b,c,d,e} \mathrm{,} \: \mathrm{param} \right]
% \end{equation}

% % Process model
% \begin{equation}
%   \label{eq:model-process}
%   \mu_{\boldsymbol{s}, \boldsymbol{t},l} = 
%     N_{\boldsymbol{s}, \boldsymbol{t},l} \left(1 - \mathrm{exp} \! \left[ - \lambda_{s,l} \,
%     \omega_{s, k[t]} F_{s,\mathrm{year}\left[ t \right]} \right] \right) W_{s}
% \end{equation}

% Process model
\begin{equation}
  \label{eq:model-process}
  \boldsymbol{\mu}_{\boldsymbol{t},l} = \boldsymbol{N}_{\boldsymbol{t},l} \, \boldsymbol{\Psi}_{t,l}
\end{equation}

% % Numbers model
% \begin{equation}
%   \label{eq:model-numbers}
%   \boldsymbol{N}_{t_{0}, t, l} = \boldsymbol{N}_{t_{0}, t-1, l} \:
%     \mathrm{diag} \! \left[ - \mathrm{exp} \! \left[ - \boldsymbol{Z}_{t-1,l} \right]  \right]
%     \boldsymbol{\Gamma}_{\mathrm{block}[t-1],l} \mathrm{,} \quad t > t_0
% \end{equation}

% Numbers model
\begin{equation}
  \label{eq:model-numbers}
  \boldsymbol{N}_{t_{0}, t+1, l} = \boldsymbol{N}_{t_{0}, t, l} \,
    \boldsymbol{\Lambda}_{t,l} \,
    \boldsymbol{\Gamma}_{\mathrm{block}[t],l} \mathrm{,} \quad t \geq t_0
\end{equation}

% Reporting model
\begin{equation}
  \label{eq:model-reporting}
  \boldsymbol{\Psi}_{t,l} = \left( \boldsymbol{I} - \mathrm{exp} \! \left[ -
    \boldsymbol{\lambda}_{l} \boldsymbol{\omega}_{k[t]}  
    \boldsymbol{F}_{\mathrm{year}[t]} \right] \right) \boldsymbol{W}
\end{equation}

% % Reporting model
% \begin{equation}
%   \label{eq:model-reporting}
%   \boldsymbol{\Psi}_{t,l} = \mathrm{diag} \! \left[ \left( 1 - \mathrm{exp} \! \left[ -
%     \boldsymbol{\lambda}_{l} \odot \boldsymbol{\omega}_{k[t]} \odot 
%     \boldsymbol{F}_{\mathrm{year}[t]} \right] \right) \odot \boldsymbol{W} \right]
% \end{equation}

% % Mortality model
% \begin{equation}
%   \label{eq:model-mortality}
%   \boldsymbol{Z}_{t,l} = \boldsymbol{\lambda}_{l} \odot \boldsymbol{\omega}_{k[t]} \odot 
%     \boldsymbol{F}_{\mathrm{year}[t]} + \frac{1}{K} \boldsymbol{M} + \frac{\eta}{K}
% \end{equation}

% % Mortality model
% \begin{equation}
%   \label{eq:model-mortality}
%   Z_{s,t,l} = \lambda_{s,l} \omega_{s,k[t]} 
%     F_{s,\mathrm{year}[t]} + \frac{1}{K} M_{s} + \frac{\eta}{K}
% \end{equation}

% Survival model
\begin{equation}
  \label{eq:model-survival}
    \boldsymbol{\Lambda}_{t,l} = 
    - \mathrm{exp} \! \left[ -
    \boldsymbol{\lambda}_{l} \boldsymbol{\omega}_{k[t]}  
    \boldsymbol{F}_{\mathrm{year}[t]} + \frac{1}{K} \boldsymbol{M} + \frac{\eta}{K} \right]
\end{equation}

% % Survival model
% \begin{equation}
%   \label{eq:model-survival}
%     \boldsymbol{\Lambda}_{t,l} = 
%     \mathrm{diag} \! \left[ - \mathrm{exp} \! \left[ -
%     \boldsymbol{\lambda}_{l} \odot \boldsymbol{\omega}_{k[t]} \odot 
%     \boldsymbol{F}_{\mathrm{year}[t]} + \frac{1}{K} \boldsymbol{M} + \frac{\eta}{K} \right] \right]
% \end{equation}

% Release model
\begin{equation}
  \label{eq:model-release}
  N_{\boldsymbol{s}_{0},\boldsymbol{t}_{0},l} \sim 
    \mathrm{Binom} \! \left[ x_{\boldsymbol{s}_{0},t_{0},l} \mathrm{,} \: \left( 1 - \nu \right) \right]
\end{equation}

% Movement model
\begin{equation}
  \label{eq:model-movement}
  \boldsymbol{P}_{\mathrm{block}[t],l} = \boldsymbol{\Gamma}^{K}_{\mathrm{block}[t],l}
\end{equation}

% % TODO Move to table
% % Seasons model
% \begin{equation}
%   \label{eq:model-seasons}
%   K = 6
% \end{equation}

% % Move to text
% % Variance model
% \begin{equation}
%   \label{eq:model-variance}
%   \mathrm{Var} \! \left[ y  \right] = \mu + \frac{\mu^2}{\phi}
% \end{equation}

\subsection{Priors}

% Prior phi
\begin{equation}
  \label{eq:prior-dispersion}
  \phi \sim \mathrm{Distrn} \! \left[ \mathrm{param} \right]
\end{equation}

% Prior selectivity
\begin{equation}
  \label{eq:prior-selectivity}
  \lambda_{s,l} \sim \mathrm{Beta} \! \left[ \mathrm{val, val} \right]
\end{equation}

% Prior fishing weight
\begin{equation}
  \label{eq:prior-weight}
  \mathrm{diag} \! \left[ \boldsymbol{\omega}_{k[t]} \right] \sim 
    \mathrm{MultiDistrn} \! \left[  \mathrm{val, val} \right]
\end{equation}

% Prior fishing
\begin{equation}
  \label{eq:prior-fishing}
  F_{s,\mathrm{year}} \sim 
    \mathrm{Distrn} \! \left[ F^{\mathrm{assess}}_{s,\mathrm{year}} \mathrm{, val} \right]
\end{equation}

% Prior reporting
\begin{equation}
  \label{eq:prior-reporting}
  W_{s} \sim \mathrm{Beta} \! \left[ \mathrm{val, val} \right]
\end{equation}

% Prior mortality
\begin{equation}
  \label{eq:prior-mortality}
  M_{s}  \sim \mathrm{Distrn} \! \left[ \mathrm{param} \right]
\end{equation}

% Prior tagloss
\begin{equation}
  \label{eq:prior-tagloss}
  \eta \sim \mathrm{Distrn} \! \left[ \mathrm{val, val} \right]
\end{equation}

% Prior initloss
\begin{equation}
  \label{eq:prior-initloss}
  \nu \sim \mathrm{Beta} \! \left[ \mathrm{val, val}  \right]
\end{equation}

\newpage
\section{Tables}

% Model indexes
\begin{table}[ht]
  \centering
  \caption{Model indexes}
  \renewcommand\arraystretch{1.2}
  \label{tab:model-indexes}
  \begin{tabular}{l l l l r}
    \toprule
    \textbf{Symbol} & \textbf{Value} & \textbf{Definition} & \textbf{Type} \\
    \toprule
    % Index limits
    $S$ & $= 3$ or $6$ & Number of spatial regions & Limit \\
    $T$ & $\in \mathbb{Z}^{+}$ & Number of model time steps & Limit \\
    $K$ & $= 6$ & Number of time steps per year & Limit \\
    $L$ & $= 1$ or $2$ & Number of release size classes & Limit \\
    \midrule
    % Index variables
    $\boldsymbol{s}$ & $= (s_0, s)$ & Spatial index vector & Index \\
    $\boldsymbol{t}$ & $= (t_0, t)$ & Temporal index vector & Index \\
    $\boldsymbol{s}_0$ & $= (s_0, s_0)$ & Initial spatial index vector & Index \\
    $\boldsymbol{t}_0$ & $= (t_0, t_0)$ & Initial temporal index vector & Index \\
    \midrule
    $s$ & $\in \left[1 \, .. \, S \right]$ & Spatial region & Index \\
    $t$ & $\in \left[1 \, .. \, T \right]$ & Time step & Index \\
    $s_0$ & $\in \left[1 \, .. \, S \right]$ & Initial spatial region & Index \\
    $t_0$ & $\in \left[1 \, .. \, T \! - \! 1 \right]$ & Initial model time step & Index \\
    $k$ & $\in \left[1 \, .. \, K \right]$ & Time step within year & Index \\
    $l$ & $\in \left[1 \, .. \, L \right]$ & Release size class & Index \\
    \bottomrule
  \end{tabular}
\end{table}

% Data
\begin{table}[ht]
  \centering
  \caption{Data}
  \renewcommand\arraystretch{1.2}
  \label{tab:data}
  \begin{tabular}{l l l l r}
    \toprule
    \textbf{Symbol} & \textbf{Value} & \textbf{Dim} & \textbf{Definition} & \textbf{Type} \\
    \toprule
    $\boldsymbol{y}_{\boldsymbol{t},l}$ & $\in \mathbb{Z}^{+,0}$ & $[S,S]$ & Sablefish tags recovered & Count \\
    $\boldsymbol{x}_{t_0,l}$ & $\in \mathbb{Z}^{+,0}$ & $[S,S]$ & Sablefish tags released & Count \\
    $\boldsymbol{F}^{\mathrm{assess}}_{\mathrm{year}}$ & $\in \mathbb{R}^{+,0}$ & $[S,S]$ & Annual sablefish fishing mortality rate & Inst. \\    
    \bottomrule
  \end{tabular}
\end{table}

% Model derived quantities
\begin{table}[ht]
  \centering
  \caption{Derived quantities}
  \renewcommand\arraystretch{1.2}
  \label{tab:model-derived}
  \begin{tabular}{l l l l r}
    \toprule
    \textbf{Symbol} & \textbf{Value} & \textbf{Dim} & \textbf{Definition}  & \textbf{Type} \\
    \toprule
    $\boldsymbol{N}_{\boldsymbol{t},l}$ & $\in \mathbb{R}^{+,0}$ & $[S,S]$ & Sablefish tag abundance & Numbers \\
    $\boldsymbol{\mu}_{\boldsymbol{t},l}$ & $\in \mathbb{R}^{+,0}$ & $[S,S]$ & Expected tags recovered & Numbers \\
    $\boldsymbol{\Psi}_{t,l}$ & $\in \left[0, 1 \right]$ & $[S,S]$ & Sablefish tag recovery rate & Stepwise \\
    $\boldsymbol{\Lambda}_{t,l}$ & $\in \left[0, 1 \right]$ & $[S,S]$ & Sablefish tag survival rate & Stepwise \\    
    $\boldsymbol{P}_{\mathrm{block},l}$ & $\in \left[0, 1 \right]$ & $[S,S]$ & Sablefish movement rate & Yearly \\
    \bottomrule
  \end{tabular}
\end{table}

% Model parameters
\begin{table}[ht]
  \centering
  \caption{Model parameters}
  \renewcommand\arraystretch{1.2}
  \label{tab:model-parameters}
  \begin{tabular}{l l l l r}
    \toprule
    \textbf{Symbol} & \textbf{Value} & \textbf{Dim} & \textbf{Definition} & \textbf{Type} \\
    \toprule
    $\boldsymbol{\Gamma}_{\mathrm{block},l}$ & $\in \left[0, 1 \right]$ & $[S,S]$ & Sablefish movement rate & Stepwise \\
    $\boldsymbol{\lambda}_{l}$ & $\in \left[0, 1 \right]$ & $[S,S]$ & Sablefish fisheries size selectivity & Per fish \\    
    $\boldsymbol{\omega}_{k}$ &  $\in \left[0, 1 \right]$ & $[S,S]$ & Seasonal fishing mortality weight & Stepwise \\
    $\boldsymbol{F}_{\mathrm{year}}$ & $\in \mathbb{R}^{+,0}$ & $[S,S]$ & Annual sablefish fishing mortality rate & Inst. \\
    $\boldsymbol{M}$ & $\in \mathbb{R}^{+,0}$ & $[S,S]$ & Annual sablefish natural mortality rate & Inst. \\
    $\boldsymbol{W}$ & $\in \left[0, 1 \right]$ & $[S,S]$ & Sablefish tag reporting rate & Per tag \\        
    $\eta$ & $\in \mathbb{R}^{+,0}$ & Scalar & Annual sablefish tag loss rate & Inst. \\    
    $\nu$ & $\in \left[0, 1 \right]$ & Scalar & Sablefish initial tag loss rate & Per tag \\
    $\phi$ & $\in \mathbb{R}^{+}$ & Scalar & Negative binomial dispersion & NB2 \\
    \bottomrule
  \end{tabular}
\end{table}

% End
\end{document}

%%% Local Variables:
%%% mode: LaTeX
%%% TeX-master: t
%%% TeX-master: t
%%% TeX-master: t
%%% TeX-master: t
%%% End:
